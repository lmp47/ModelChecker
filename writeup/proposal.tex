% A project proposal.
% The format draws heavily on the example project proposal
% and the Model Project Proposals (and their LaTeX sources) available
% http://www.cl.cam.ac.uk/teaching/projects/

\documentclass[12pt,a4paper,twoside]{article}
\usepackage[pdfborder={0 0 0}]{hyperref}
\usepackage[margin=25mm]{geometry}
\usepackage{graphicx}
\begin{document}

\vfil

\centerline{\Large Part II Project Proposal}
\vspace{0.4in}
\centerline{\Large \bf A Model Checker Using IC3}
\vspace{0.4in}
\centerline{\large Lauren Pick, Homerton College}
\vspace{0.4in}
\centerline{\large 23 October 2015}
\vspace{0.4in}

\noindent
{\bf Project Originator:} Lauren Pick
\vspace{0.2in}

\noindent
{\bf Project Supervisors:} Dr Dominic Mulligan, Dr Ali Sezgin
\vspace{0.2in}

\noindent
{\bf Supporting Supervisor:} Dr Alan Mycroft
\vspace{0.2in}

\noindent
{\bf Director of Studies:} Dr Bogdan Roman
\vspace{0.2in}
 
\noindent
{\bf Project Overseers:} Dr Richard Gibbens, Dr Larry Paulson


\section*{Introduction and Description of the Work}
Model checking is one way of assessing whether or not a hardware or
software system has certain properties. For example, model checkers can be
used to check systems for safety properties by finding examples of states
that violate the properties or by proving that all states have the
properties.

Explicit-state model checking can be infeasible for systems with a large
number of states, but symbolic model checking, which represents
states and the transition relation between them as boolean expressions,
can handle more states. Symbolic model checking initially relied on the
efficient representation of boolean expressions through binary decision
diagrams (BDDs), but BDDs can still consume a large amount of space, and
finding an ordering for BDD variables that keeps the BDDs small can become
costly \cite{biere99}.

Symbolic model checking techniques that rely on SAT solvers provide an
alternative to BDD-based approaches. SAT-based approaches include
bounded model checking \cite{biere99} and $k$-induction \cite{demoura03},
but both of these approaches involve unrolling the transition relation,
which can lead to long SAT solver queries.

IC3 \cite{bradley11} is a more recently developed SAT-based algorithm for
the symbolic model checking of safety properties. Instead of unrolling the
transition relation and considering entire paths, IC3 maintains a set of
frames $F_0,...,F_k$, where each frame $F_i$ is an overapproximation of
the set of states reachable in at most $i$ steps, and considers at most one
step of the transition relation from a particular frame at a time.
As a result, IC3 can find inductive strengthenings that tend
to be smaller and more convenient than those found by BMC-based
techniques such as $k$-induction, which finds strengthenings that are the
negations of spurious counterexample paths \cite{demoura03}, and the SAT
queries that IC3 makes tend to be simpler \cite{bradley12}.

This project focuses on implementing a symbolic model checker for verifying
safety properties of hardware. The model checker will include a new
implementation of the IC3 algorithm in Haskell, which will make use of an
existing SAT solver.

\section*{Starting Point}

I begin the project with some experience programming in Haskell
from a summer internship and no experience with model checking or using
a SAT solver. I have informally acquired some knowledge about model checking
to formulate this project idea.

\section*{Substance and Structure of the Project}

The project aims to implement a hardware model checker that takes its
inputs in AIGER format and queries the MiniSat SAT solver.

\noindent The structure of the project can be broken down into the following
components:

\begin{enumerate}

\item {\bf Parsing AIGER format} The model checker takes its inputs in AIGER
format and will, as a result, require an AIGER parser. The AIGER format
is fairly simple and hand-coding a parser for it should be suitable.

\item {\bf Interfacing with MiniSat} The model checker
will be using the MiniSat SAT solver, so an API that allows the
model checker to query MiniSat will be required.

\item {\bf Implementing the IC3 algorithm} 
The main aspect of the project is the implementation of the IC3 algorithm.
The implementation will largely be based on the algorithm as described in
\cite{bradley11,bradley12}. 

\item {\bf Evaluating the model checker}
The model checker will be evaluated by measuring its performance on
checking examples. Though the project does not focus greatly on the
efficiency of the implementation, it may still be interesting to
see how the performance of this IC3 implementation in Haskell compares with
other implementations. As a result, benchmarks taken for the
model checker will be compared with further benchmarks taken for
Aaron Bradley's reference IC3 implementation, which is
implemented in C++.
Given that the reference implementation takes its inputs in AIGER format
and also uses MiniSat, the benchmarks should provide a means to compare
the IC3 implementations specifically.

\item {\bf Writing the dissertation}

\end{enumerate}

\subsection*{Possible Extensions}
If the aforementioned aspects of the project are completed, carrying out
the following extensions could be possible:
\begin{itemize}
\item Interfacing with other SAT solvers, and possibly performing additional
benchmarking; comparing the performance of the model checker when used with
different SAT solvers may be of interest since the performance
of IC3 implementations tend to vary considerably depending on the
characteristics of the underlying SAT solver.
\item Model checking properties of real hardware as a case study.
\item Implementing abstraction-refinement as described in \cite{vizel12}.
\end{itemize}


\section*{Success Criteria}

The project will be a success if the following have been completed:
\begin{itemize}
\item The AIGER parser has been implemented.
\item The MiniSat interface has been implemented.
\item The IC3 algorithm has been implemented.
\item The model checker should be able to solve some small examples.
\end{itemize}


\section*{Timetable: Workplan and Milestones}

\begin{enumerate}

\item {\bf 16 October 2015 -- 28 October 2015}

Preliminary reading. Get familiar with the AIGER format, MiniSat and
relevant Haskell libraries and tools for implementing the components
of the project.

\item {\bf 29 October 2015 -- 4 November 2015}

Write an AIGER format parser.
\\
Milestone: Parser completed. Relevant information from
AIGER files can be extracted.

\item {\bf 5 November 2015 -- 18 November 2015}

Implement a MiniSat interface.
\\
Milestone: MiniSat interface completed, enabling the model checker to use
MiniSat to solve SAT problems.

\item {\bf 19 November 2015}

Begin implementing the IC3 algorithm.

\item {\bf Michaelmas vacation}

Continue implementing the IC3 algorithm.

\item {\bf 14 January 2016 -- 27 January 2016}

Write progress report. Finish implementation of the IC3 algorithm.
\\
Milestones: Progress report completed.
Working implementation of the model checker completed.

\item {\bf 28 January 2016 -- 10 February 2016}

Measure and compare this IC3 implementation's performance and the reference
implementation's performance.
\\
Milestone: Evaluation completed.

\item {\bf 11 February 2016 -- 11 March 2016}

Write the main parts of the dissertation.
\\
Milestone:
Finished writing main parts of dissertation: introduction, preparation,
implementation and evaluation chapters.

\item {\bf Easter vacation}

If necessary, use this time for catching up.
Otherwise, work on extensions, starting with interfacing with other
SAT solvers.
Finish writing dissertation.
\\
Milestones: All implementation and evaluation completed.
Draft dissertation completed.

\item {\bf 21 April 2016 -- 4 May 2016}

Proofread and edit dissertation as necessary.
\\
Milestone: Dissertation ready for submission.

\item {\bf 5 May 2016 -- 13 May 2016}

Time left for catching up in case any delays have occurred in the
completion of any milestones.
\\
Milestone: Dissertation submitted.

\end{enumerate}

\section*{Resources Required}

For the project I will mostly make use of my laptop, which runs OS X 10.8.
I accept full responsibility for this machine and I have made contingency
plans to protect myself against hardware and/or software failure. 
If my main computer fails, I will use MCS computers.
I will use GitHub for backup and git for revision control.

\noindent I will also be using:
\begin{itemize}
\item AIGER utilities, available \url{http://fmv.jku.at/aiger/}
\item MiniSat, available \url{https://github.com/niklasso/minisat}
\item Models from the Hardware Model Checking Competition, such as those
available \url{http://fmv.jku.at/hwmcc10/}
\item Aaron Bradley's Reference IC3 implementation, available
\url{https://github.com/arbrad/IC3ref}
\end{itemize}

\bibliographystyle{plain}
\bibliography{refs}
\end{document}
