% Template for a Computer Science Tripos Part II project dissertation
\documentclass[12pt,a4paper,twoside,openright]{report}
\usepackage[pdfborder={0 0 0}]{hyperref}    % turns references into hyperlinks
\usepackage[margin=25mm]{geometry}  % adjusts page layout
\usepackage{graphicx}  % allows inclusion of PDF, PNG and JPG images
\usepackage{verbatim}
\usepackage{amssymb}
\usepackage{docmute}   % only needed to allow inclusion of proposal.tex

\raggedbottom                           % try to avoid widows and orphans
\sloppy
\clubpenalty1000%
\widowpenalty1000%

\renewcommand{\baselinestretch}{1.1}    % adjust line spacing to make
                                        % more readable

\begin{document}

\bibliographystyle{plain}


%%%%%%%%%%%%%%%%%%%%%%%%%%%%%%%%%%%%%%%%%%%%%%%%%%%%%%%%%%%%%%%%%%%%%%%%
% Title


\pagestyle{empty}

\rightline{\LARGE \textbf{Lauren Pick}}

\vspace*{60mm}
\begin{center}
\Huge
\textbf{A Model Checker Using IC3} \\[5mm]
Computer Science Tripos -- Part II \\[5mm]
Homerton College \\[5mm]
\today  % today's date
\end{center}

%%%%%%%%%%%%%%%%%%%%%%%%%%%%%%%%%%%%%%%%%%%%%%%%%%%%%%%%%%%%%%%%%%%%%%%%%%%%%%
% Proforma, table of contents and list of figures

\pagestyle{plain}

\chapter*{Proforma}

{\large
\begin{tabular}{ll}
Name:                  & \bf Lauren Pick                           \\
College:               & \bf Homerton College                      \\
Project Title:         & \bf A Model Checker Using IC3             \\
Examination:        & \bf Computer Science Tripos -- Part II, 2016 \\
Word Count:            & \bf                                       \\
Project Originator:    & Lauren Pick                               \\
Supervisors:           & Dr Dominic Mulligan, Dr Ali Sezgin        \\ 
Supporting Supervisor: & Prof Alan Mycroft
\end{tabular}
}


\section*{Original Aims of the Project}

Describe the original aims of the project, i.e. summarize information from
the ``Substance and Structure'' and ``Success Criteria'' sections of the
project proposal.

\section*{Work Completed}

Describe the work completed as part of project, including extensions.

\section*{Special Difficulties}

None.
 
\newpage
\section*{Declaration}

I, Lauren Pick of Homerton College, being a candidate for Part II of the Computer
Science Tripos, hereby declare
that this dissertation and the work described in it are my own work,
unaided except as may be specified below, and that the dissertation
does not contain material that has already been used to any substantial
extent for a comparable purpose.

\bigskip
\leftline{Signed [signature]}

\medskip
\leftline{Date [date]}

\tableofcontents

%\listoffigures

%\newpage
%\section*{Acknowledgements}


%%%%%%%%%%%%%%%%%%%%%%%%%%%%%%%%%%%%%%%%%%%%%%%%%%%%%%%%%%%%%%%%%%%%%%%
% now for the chapters

\pagestyle{headings}

\chapter{Introduction}

%Provide a brief description of model checking in general.

Model checking attempts to verify that a hardware or
software system has certain properties. Given a model of the system
and a specified property, a model checker will automatically check
whether all states in the system satisfy this property.

The initial approach to the model checking problem was explicit-state
model checking, which involves considering all reachable states in the
model. All model-checking approaches suffer from limitations on the size
of the systems they can model check in practice as a result of
the state explosion problem: the number of
states in a system can be (and often is) exponential in the
number of state variables \cite{Clarke2012}. 
%\cite{McMillan1992, Clarke2012}.

Symbolic model checking arose as a method of mitigating the effects of
the state explosion problem to some extent. Symbolic model checking
represents states and the transition relation between them as boolean
expression, allowing sets of states to be represented efficiently
as boolean expressions involving state variables as opposed to an
explicit list of each individual state in the set. 
%\cite{McMillan1992, Clarke2012}.

A brief description symbolic model-checking and SAT-based model-checking
follows to provide further context for this project, which focuses on implementing
the IC3 algorithm, a SAT-based model-checking algorithm.


\section{Symbolic Model Checking}

%Discuss BDD-based and SAT-based model checking techniques.

Symbolic model checking was originally invented for use with ordered
binary decision diagrams (BDDs), which provide an efficient representation
of boolean formulas using canonical forms for the formulas.
The efficiency of BDDs in storing boolean formulas allowed for the model
checking of systems with larger numbers of states than could be handled
by explicit-state model checking. % See McMillan 1992
% Transition into talking about SAT-based model checking
The efficiency of BDD representations relies on choosing an appropriate
ordering, which can be computationally expensive, and in some cases,
there is no such ordering that results in a space-efficient BDD.
% See Biere, Cimatti, Clarke, Zhu 1999

An alternative to BDD-based symbolic model-checking techniques are SAT-based
techniques, which do not use the canonical representations of boolean formulas as
BDDs and make use of SAT procedures. Such techniques include bounded
model checking (BMC), which proves properties by finding counterexamples of certain
lengths; $k$-induction, which proves properties inductively; and the IC3 algorithm
that is the focus of this project. A brief description and comparison of these techniques
follows.
% See Biere, Cimatti, Clarke, Zhu 1999
% k-induction: Sheeran, Singh, Stalmarck 2000
% Mention FSIS?
% Mention ITP?

Many modern SAT-based model checkers are based on BMC, which begins at the
initial state of the transition system and searches for paths of length
$k$ from the initial state that violate the negated property. If no path of length $k$
is found, BMC increments $k$ and searches again. BMC is effective at finding
counterexamples, but for some large systems, BMC is incomplete unless it is allowed
to reach the point at which $k$ is maximal path length.

The $k$-induction algorithm is similar to BMC in its unrolling of the transition relation
to consider paths of length $k$, but it also incorporates induction. At each depth
$k$, the algorithm asserts that the desired property holds at each state before the
final state. % Understanding IC3

\section{The IC3 Algorithm}

%Give context for the IC3/PDR algorithm, e.g. history and comparisons
%with other SAT-based model checking algorithms such as BMC and $k$-induction.

The IC3 algorithm (also called PDR) is a SAT-based model-checking algorithm
for proving the safety properties (i.e. properties which must hold in all reachable states)
of hardware. The first implementation of the algorithm \verb,ic3, placed third in HWMCC'10,
and since then, several variants of the algorithm have been developed.
%PDR paper, 2015 paper

As in $k$-induction, properties are proved through induction: the algorithm considers
reachable sets of states $k$ steps away from the initial state until reaching a fixed point.
% Note: find when k-induction started using counterexamples
As with later variants of $k$-induction, the IC3 algorithm also discovers new invariants
if the initial assumptions are not strong enough to prove the desired property, but unlike
$k$-induction, the safety property guides the discovery of the invariants. As a result,
the discovered invariants are more relevant for proving the safety property.

Furthermore, IC3 does not unroll the transition relation as $k$-induction or other
BMC-based methods, but instead leading to smaller, simpler SAT queries. As
a result, IC3 requires less memory than such methods.

% See Understanding IC3

\section{Further Work}


%Mention other applications of the IC3 algorithm, e.g. to checking
%LTL properties and to software model checking.

%Mention IIV?

While IC3 algorithm is for model checking safety properties of hardware, there are
applications of the algorithm in model checking LTL and CTL properties and
model checking software.

The FAIR model-checking algorithm, which model-checks $\omega$-regular properties,
makes use of a safety model checker such as IC3, and IICTL, which model-checks
CTL formulas, makes use of both a safety model checker such as IC3 as well as a fair cycle
finder such as FAIR.
% See Understanding IC3, IICTL, 

Other work generalizes IC3 to deal with infinite-state systems. This generalization of IC3
uses an underlying SMT solver rather than a SAT solver and has been used to check
control-flow graphs of programs. More recently, Johannes Birgmeier, Aaron Bradley,
and Georg Weissenbacher have introduced CTIGAR, a method of abstraction-refinement
based on IC3's counterexamples to induction rather than the counterexamples in CEGAR,
which experiments suggest is competitive with CEGAR-based techniques for software
verification.
%Software Model Checking via IC3, CTIGAR, etc.

\section{This Project}

This project's aim was to implement a basic form of the IC3 algorithm in Haskell that can
model check several small examples.

In addition to the single basic form of the algorithm, I also developed several
variants of the algorithm.

\chapter{Preparation}

\section{Requirements Analysis}

\subsection{Project-specific Requirements}
%Describe the requirements for the project: AIGER parser, Minisat interface,
%transition system representation, algorithm implementation.

A SAT-based model checker requires a way of taking input models and also requires a
way to solve SAT queries. I chose the AIGER format for representing the hardware models
and the MiniSat SAT solver for answering SAT queries, resulting in a need for an AIGER
parser and an interface to MiniSat in Haskell.

Given that the model checking algorithm deals with transition systems (discussed later),
the implementation also requires a representation of transition systems, which should
correspond to the input hardware model. A further requirement is the implementation of the
IC3 algorithm itself.

In summary, the project-specific requirements are as follows:
\begin{itemize}
\item AIGER parser
\item MiniSat interface
\item Transition system representation
\item IC3 algorithm implementation
\end{itemize}

\subsection{General Requirements}
%Also, version control and testing.

\section{Tools Used}
Mention the tools used (and what they were used for), including Cabal,
HUnit, {\tt hsc2hs}, Minisat, {\tt bliftoaig}, Criterion.
Not too sure about subsections.

While developing the project, Git was used for version control.
Cabal was used for package/dependency management.
HUnit was used for testing.
Criterion was used for benchmarking.

\section{Model Specification}
%Describe the AIGER (old and new version) and BLIF formats and
%{\tt bliftoaig} from the AIGER utilities.

Both the AIGER format and Berkeley Logic Interchange Format (BLIF) were used to specify
hardware models.
Hardware models that the model checker accepts as input
are specified using the AIGER format, which is the format used to specify
hardware models in the Hardware Model Checking Competitions; however using the
AIGER format to specify larger models was cumbersome, so such models were specified
using BLIF and converted to AIGER format using the AIGER library's \verb,bliftoaig, tool.

\subsection{AIGER}

The AIGER format has both an ASCII and binary version, where the ASCII format is more flexible and human readable,
and the binary version is more compact. The Hardware Model Checking Competition's examples
are in the latter of the two.

A new version of the AIGER format is currently under development, with examples from HWMCC'14
using the new version.

\subsubsection{Old version}

All AIGER files begin with a header of the form
\begin{verbatim}
V M I L O A
\end{verbatim}
where:
\begin{itemize}
\item \verb,V, can take on values \verb,aag,, specifying that the file is in the ASCII format or \verb,aig,,
specifying that the file is in the binary format.
\item \verb,M, gives the maximum index of a variable
\item \verb,L, gives the number of latches
\item \verb,O, gives the number of outputs
\item \verb,A, gives the number of AND gates
\end{itemize}

Variables are represented in AIGER with indices greater than 1, with 0 indexing the constant
value {\it False} and 1 indexing {\it True}.

In the ASCII version of the format,

\subsection{Berkeley Logic Interchange Format}

\section{The IC3 Algorithm}
%Describe how the IC3 algorithm works when trying to prove a model has a property
%$P$.
Given a hardware model (i.e. a finite-state transition system) and a safety property $P$, IC3
aims either to prove inductively that $P$ holds at all reachable states from the initial state or
to find a $\neg P$ state that can be reached.

\subsection{Transition Systems}
A transition system is a tuple $(i,x,I,T)$ consisting of a set of input variables $i$, state variables $x$,
next-state variables $x'$, an initial set of states represented by the boolean expression $I(x)$ and
a transition relation represented by the boolean expression $T(i,x,x')$.

A single state of the system (or a singleton set containing that state) is specified through the assignment
of all state variables in $x$.

\subsection{Frames}
As with other inductive approaches to model checking,
the IC3 algorithm maintains a set of $k$ frames $F_0,\ldots,F_k$, where
each frame $F_i$ is a set of clauses that represent an overapproximation of the
set of states that reachable by the transition system in $i$ steps
(so $F_0$ is just the initial state set $I$).

If $F_i = F_{i + 1}$ holds for any $i$ at any point, then a fixed point has
been found, and the algorithm terminates.

\subsection{Algorithm}
%Initiation query
The initiation query $I \Rightarrow P$ is used to check that the property
holds in the initial state $I$.

%Consecution query
The consecution query for a frame $F_k$ is used to check whether the property $P$
necessarily holds in the next frame.
$$F_k \wedge T \Rightarrow P'$$

%Recursive part of algorithm
The algorithm extends its set of frames $F_0,\ldots,F_k$ with a new frame $F_{k + 1}$
by running the consecution query.
If the consecution query is successful, then the new frame $F_{k + 1}$ with clause
$\{P\}$ can be added.
All clauses in frame $F_k$ are then propagated to $F_{k + 1}$:
For each clause $C \in F_k$, if $F_k \wedge T \Rightarrow C'$ holds, then $C'$
is added to $F_{k + 1}$.

%Describe what happens when a consecution query fails
If a consecution query $F_k \wedge T \Rightarrow P'$ fails, then that means that
there is an $F_k$ state that is a predecessor of the $\neg P'$ state,
i.e. there is an $F_k$ state $s$ and a $\neg P'$ state v with $T(i,s,v')$. The state
$s$ is a \emph{counterexample to induction} state.

Because $F_k$ is an overapproximation, the algorithm aims to refine $F_k$
by finding a clause $c$ that holds at depth $k$ such that
$F_k \wedge c \wedge T \Rightarrow P'$ holds:

Such a $c$ must be such that for the counterexample to induction $s$,
$$c \Rightarrow \neg s,$$
with an obvious choice being $c = \neg s$, which can be found easily using
a SAT solver. In practice, it is better to choose a smaller $c$, which can allow
for a set of several states to be eliminated in $F_k$ rather than just eliminating
$s$.

\chapter{Implementation}
Briefly mention sections in the chapter and the example models
written in AIGER and BLIF formats.

\section{AIGER Parser}
Discuss the implementation of the AIGER parser. In particular, mention
the handling of both the older and newer AIGER format versions for both
the ASCII and binary versions of the format and the representation
of AIG models.

\section{Minisat Interface}
%Describe the process of implementing the Haskell bindings for Minisat
%functions, including the wrapper functions for Minisat written in C.

MiniSat serves as the SAT solver for this implementation of the IC3 algorithm.
Because the Haskell Foreign Function Interface cannot interface with C++ directly,
the interface to the MiniSat SAT solver is composed of a C wrapper for the relevant
MiniSat functions and classes and a Haskell interface to the C wrapper.

\subsection{Relevant MiniSat Functions}

To solve a SAT query, MiniSat creates an instance of a \verb,Solver, object,
which contains a set of variables, sets of clauses represented by \verb,VecLit, instances
that must be satisfied each SAT query, a model and a conflict vector.
The \verb,solveWithAssumps, function makes a SAT query that must satisfy all of the clauses
in the \verb,Solver, as well as all the literals supplied in the assumption \verb,VecLit,.

If there has been at least one query made of the \verb,Solver, object, and the query was
satisfiable, the \verb,Solver,'s \verb,model, variable points to a set of variable assignments
for that SAT query.
If there has been at least one query made of the \verb,Solver, object, and the query was
unsatisfiable, the \verb,Solver,'s \verb,conflict, variable points to a set of literals that
contains the assumed literals that caused the query to be unsatisfiable.

This model checker uses instances of \verb,SimpSolver,, a subclass of the \verb,Solver, class
that does simplification and returns full assignments.

\subsection{C Wrapper}

Much of the C wrapper is straightforward: every MiniSat class is replaced with a C type,
and every MiniSat function is replaced with a function with an \verb,extern C, function that
calls the MiniSat C++ function.
I also added an additional \verb,result, struct to allow for the results of a SAT query to
be returned from a single function call. The struct contains not only the result of the SAT
query, but also pointers to the model and conflict vector (if any) of the \verb,Solver,.
The wrapper function for \verb,solveWithAssumps, returns a pointer to a \verb,result, struct
rather than just whether or not the query was satisfiable.

\subsection{Haskell Interface}

The Haskell interface makes use of the Haskell FFI as well as the \verb,hsc2hs, preprocessor
for handling the \verb,result, struct.

Using just the Haskell FFI for calling the C functions is typical, does not provide a sufficient
abstraction for use by the rest of the model checker. I wrote further functions to allow for a more 
natural interface to MiniSat, making use of \verb,unsafePerformIO, to have the functions return
values outside the \verb,IO, monad.

Many of the functions and datatypes in the interface are analogous to functions and structs in
the C wrapper and C++ implementation of MiniSat. For example, the \verb,Solver, datatype is an
analogue to the MiniSat \verb,Solver, object, and itself contains a pointer to an instance of
a MiniSat \verb,Solver, object. Similarly, functions such as \verb,solveWithAssumps, work
analogously to the C wrapper's \verb,solveWithAssumps,, returning a \verb,Result, that contains
whether or not the query was satisfiable and the model or conflict vector (if any).

The information kept in a \verb,Result, is taken directly from the \verb,result, returned by
the C Wrapper functions. I used the \verb,hsc2hs, preprocessor to help handle pointer offsets
when unmarshalling from the C struct. Beyond straightforward unmarshalling, some additional work
to convert from the MiniSat representation of literals to the model checker’s representation of
literals was necessary.

\section{Transition Systems}
\subsection{Representation}
Describe how transition systems and their components are represented.
\subsection{Construction}
Describe how transition systems are constructed given AIG models.

\section{Model Checking}
\subsection{Initiation}
Describe how the step involving the initiation query is implemented.
\subsection{Consecution}
Describe how the step involving the consecution query is implemented.
\subsection{Counterexamples to Induction}
Describe how the implementation discovers counterexamples
to induction and proves them unreachable.
\subsection{Improvements}
Describe improvements on simple implementation of algorithm.

\chapter{Evaluation}
\section{Performance Impact of Improvements}
Discuss performance comparison of the naive implementation of the
model checker and the final implementation of the model checker.

\section{Performance Comparison}
Discuss performance comparison of final implementation with
IC3 reference implementation, taking into account differences between
the implementations.

\chapter{Conclusion}

\section{Summary}
Summarize the project and accomplishments

\section{Further extensions}
Mention further extensions that could be implemented.


%%%%%%%%%%%%%%%%%%%%%%%%%%%%%%%%%%%%%%%%%%%%%%%%%%%%%%%%%%%%%%%%%%%%%
% the bibliography
\addcontentsline{toc}{chapter}{Bibliography}
\bibliography{refs}

%%%%%%%%%%%%%%%%%%%%%%%%%%%%%%%%%%%%%%%%%%%%%%%%%%%%%%%%%%%%%%%%%%%%%
% the appendices
\appendix

\chapter{Project Proposal}

% A project proposal.
% The format draws heavily on the example project proposal
% and the Model Project Proposals (and their LaTeX sources) available
% http://www.cl.cam.ac.uk/teaching/projects/

\documentclass[12pt,a4paper,twoside]{article}
\usepackage[pdfborder={0 0 0}]{hyperref}
\usepackage[margin=25mm]{geometry}
\usepackage{graphicx}
\begin{document}

\section*{Introduction and Description of the Work}
Model checking is one way of assessing whether or not a hardware or
software system has certain properties. For example, model checkers can be
used to check systems for safety properties by finding examples of states
that violate the properties or by proving that all states have the
properties.

Explicit-state model checking can be infeasible for systems with a large
number of states, but symbolic model checking, which represents
states and the transition relation between them as boolean expressions,
can handle more states. Symbolic model checking initially relied on the
efficient representation of boolean expressions through binary decision
diagrams (BDDs), but BDDs can still consume a large amount of space, and
finding an ordering for BDD variables that keeps the BDDs small can become
costly \cite{biere99}.

Symbolic model checking techniques that rely on SAT solvers provide an
alternative to BDD-based approaches. SAT-based approaches include
bounded model checking \cite{biere99} and $k$-induction \cite{demoura03},
but both of these approaches involve unrolling the transition relation,
which can lead to long SAT solver queries.

IC3 \cite{bradley11} is a more recently developed SAT-based algorithm for
the symbolic model checking of safety properties. Instead of unrolling the
transition relation and considering entire paths, IC3 maintains a set of
frames $F_0,...,F_k$, where each frame $F_i$ is an overapproximation of
the set of states reachable in at most $i$ steps, and considers at most one
step of the transition relation from a particular frame at a time.
As a result, IC3 can find inductive strengthenings that tend
to be smaller and more convenient than those found by BMC-based
techniques such as $k$-induction, which finds strengthenings that are the
negations of spurious counterexample paths \cite{demoura03}, and the SAT
queries that IC3 makes tend to be simpler \cite{bradley12}.

This project focuses on implementing a symbolic model checker for verifying
safety properties of hardware. The model checker will include a new
implementation of the IC3 algorithm in Haskell, which will make use of an
existing SAT solver.

\section*{Starting Point}

I begin the project with some experience programming in Haskell
from a summer internship and no experience with model checking or using
a SAT solver. I have informally acquired some knowledge about model checking
to formulate this project idea.

\section*{Substance and Structure of the Project}

The project aims to implement a hardware model checker that takes its
inputs in AIGER format and queries the MiniSat SAT solver.

\noindent The structure of the project can be broken down into the following
components:

\begin{enumerate}

\item {\bf Parsing AIGER format} The model checker takes its inputs in AIGER
format and will, as a result, require an AIGER parser. The AIGER format
is fairly simple and hand-coding a parser for it should be suitable.

\item {\bf Interfacing with MiniSat} The model checker
will be using the MiniSat SAT solver, so an API that allows the
model checker to query MiniSat will be required.

\item {\bf Implementing the IC3 algorithm} 
The main aspect of the project is the implementation of the IC3 algorithm.
The implementation will largely be based on the algorithm as described in
\cite{bradley11,bradley12}. 

\item {\bf Evaluating the model checker}
The model checker will be evaluated by measuring its performance on
checking examples. Though the project does not focus greatly on the
efficiency of the implementation, it may still be interesting to
see how the performance of this IC3 implementation in Haskell compares with
other implementations. As a result, benchmarks taken for the
model checker will be compared with further benchmarks taken for
Aaron Bradley's reference IC3 implementation, which is
implemented in C++.
Given that the reference implementation takes its inputs in AIGER format
and also uses MiniSat, the benchmarks should provide a means to compare
the IC3 implementations specifically.

\item {\bf Writing the dissertation}

\end{enumerate}

\subsection*{Possible Extensions}
If the aforementioned aspects of the project are completed, carrying out
the following extensions could be possible:
\begin{itemize}
\item Interfacing with other SAT solvers, and possibly performing additional
benchmarking; comparing the performance of the model checker when used with
different SAT solvers may be of interest since the performance
of IC3 implementations tend to vary considerably depending on the
characteristics of the underlying SAT solver.
\item Model checking properties of real hardware as a case study.
\item Implementing abstraction-refinement as described in \cite{vizel12}.
\end{itemize}


\section*{Success Criteria}

The project will be a success if the following have been completed:
\begin{itemize}
\item The AIGER parser has been implemented.
\item The MiniSat interface has been implemented.
\item The IC3 algorithm has been implemented.
\item The model checker should be able to solve some small examples.
\end{itemize}


\section*{Timetable: Workplan and Milestones}

\begin{enumerate}

\item {\bf 16 October 2015 -- 28 October 2015}

Preliminary reading. Get familiar with the AIGER format, MiniSat and
relevant Haskell libraries and tools for implementing the components
of the project.

\item {\bf 29 October 2015 -- 4 November 2015}

Write an AIGER format parser.
\\
Milestone: Parser completed. Relevant information from
AIGER files can be extracted.

\item {\bf 5 November 2015 -- 18 November 2015}

Implement a MiniSat interface.
\\
Milestone: MiniSat interface completed, enabling the model checker to use
MiniSat to solve SAT problems.

\item {\bf 19 November 2015}

Begin implementing the IC3 algorithm.

\item {\bf Michaelmas vacation}

Continue implementing the IC3 algorithm.

\item {\bf 14 January 2016 -- 27 January 2016}

Write progress report. Finish implementation of the IC3 algorithm.
\\
Milestones: Progress report completed.
Working implementation of the model checker completed.

\item {\bf 28 January 2016 -- 10 February 2016}

Measure and compare this IC3 implementation's performance and the reference
implementation's performance.
\\
Milestone: Evaluation completed.

\item {\bf 11 February 2016 -- 11 March 2016}

Write the main parts of the dissertation.
\\
Milestone:
Finished writing main parts of dissertation: introduction, preparation,
implementation and evaluation chapters.

\item {\bf Easter vacation}

If necessary, use this time for catching up.
Otherwise, work on extensions, starting with interfacing with other
SAT solvers.
Finish writing dissertation.
\\
Milestones: All implementation and evaluation completed.
Draft dissertation completed.

\item {\bf 21 April 2016 -- 4 May 2016}

Proofread and edit dissertation as necessary.
\\
Milestone: Dissertation ready for submission.

\item {\bf 5 May 2016 -- 13 May 2016}

Time left for catching up in case any delays have occurred in the
completion of any milestones.
\\
Milestone: Dissertation submitted.

\end{enumerate}

\section*{Resources Required}

For the project I will mostly make use of my laptop, which runs OS X 10.8.
I accept full responsibility for this machine and I have made contingency
plans to protect myself against hardware and/or software failure. 
If my main computer fails, I will use MCS computers.
I will use GitHub for backup and git for revision control.

\noindent I will also be using:
\begin{itemize}
\item AIGER utilities, available \url{http://fmv.jku.at/aiger/}
\item MiniSat, available \url{https://github.com/niklasso/minisat}
\item Models from the Hardware Model Checking Competition, such as those
available \url{http://fmv.jku.at/hwmcc10/}
\item Aaron Bradley's Reference IC3 implementation, available
\url{https://github.com/arbrad/IC3ref}
\end{itemize}
\end{document}


\end{document}

